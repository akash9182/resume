\documentclass[margin, centered]{res}
\topmargin=-0.5in
\oddsidemargin -.5in
\evensidemargin -.5in
\textwidth=6.5in
\itemsep=0in
\parsep=00in
\newsectionwidth{1in}
\usepackage[pdftex]{graphicx}
\usepackage{enumitem}
\usepackage{wrapfig}
\usepackage{helvet}
\usepackage[colorlinks = true,
			linkcolor = blue,
			urlcolor = blue,
			citecolor = blue,
			anchorcolor = blue]{hyperref}
\setlength{\textwidth}{6.5in}
\setlength{\textheight}{720pt}

\begin{document}

%----------------------------------------------------------------------------------------
%	NAME AND ADDRESS SECTION
%----------------------------------------------------------------------------------------\

\begin{center}
	\hspace{-\hoffset}
	\huge\bf{\href{http://akash9182.github.io/}{Akash Rana}}
\end{center}
\vspace{-7mm}
\moveleft\hoffset\vbox{\hrule width 19cm height 0.5pt}
\vspace{ 1mm}
\begin{center}
	\hspace{-\hoffset}
	\href{mailto:akash9182akash@gmail.com}{akash9182akash@gmail.com} ~\textbullet~ \(+91\) 9869729096 ~\textbullet~ \ B-32, Magha Hsg. Soc., Shristi, MiraRoad, Thane, India 
\end{center}

\vspace{-7mm}
\begin{resume}
%----------------------------------------------------------------------------------------
%	EDUCATION SECTION
%----------------------------------------------------------------------------------------
\section{Education}
\textbf{B.Tech in Electronics} \hfill 2012 - 2016 \\
\href{http://www.djsce.ac.in/}{Dwarkadas J. Sanghavi College of Engineering, Mumbai}
\begin{itemize}
 \item CGPA of {7.2}/10 
\end{itemize}
\textbf{High School} - \href{http://royalcollegemiraroad.edu.in/}{Royal College Of Arts, Science And Commerce} - 89.38\% \hfill 2010 - 2012 \\
\textbf{Secondary School} - Shanti Nagar High School, MiraRoad - 92\% \hfill 2000 - 2010

%----------------------------------------------------------------------------------------
%	EXPERIENCE SECTION
%----------------------------------------------------------------------------------------
\section{Experience}
\textbf{Software Engineer and NLP researcher, \href{www.google.com}{IOEMAS Pvt Ltd}} \hfill January, 2015 - April, 2017\\
Lead the Android Development team to deliver Chargyfi product that allows to share internet and charge devices at restaurants. Built a sentiment analysis app that scores the user sentiment as angry, neutral and happy.

\textbf{Software Engineering Intern, \href{www.google.com}{Thinkbank Solutions Pvt Ltd}} \hfill September, 2016 - December, 2016\\
Worked with Android Development team to create android application that allows to share internet with captive portal system on tablet.
Also developed GadgetBridge, an app which tracks users sleep cycle and heart rate. Worked on adding a new features that monitor users heartrate and alerts if heartrate exceeds his personalized threshold.\\


%----------------------------------------------------------------------------------------
%	TECHNICAL SKILLS SECTION
%----------------------------------------------------------------------------------------
\section{Technical \hspace{2mm} Skills}
\textbf{Strong Areas} - Machine Learning Algorithms\\
\textbf{Languages} - Python, C++, Octave, PHP, JAVA, embedded C, Verilog\\
\textbf{Tools/Frameworks} - Tensorflow, Pytorch, Keras,\LaTeX, MySQL, SQLite,Flask, Git\\
\textbf{OS} - Ubuntu and Windows 
%---------------------------------------------------------------------------------------
%	PUBLICATION SECTION
%---------------------------------------------------------------------------------------
\section{Publications}
\begin{itemize}[leftmargin=*]
\item Riyansh Karani, \textbf{Akash Rana}, Dhruv Reshamwala, Kishore Saldanha ``A floating point division unit based on Taylor-series expansion algorithm and Iterative Logarithmic Multiplie'', In Proceedings
of The Second International Conference on Computer Science, Information Technology (CSITEC -2016).\href{http://airccj.org/CSCP/vol6/csit65903.pdf}{view here}
\end{itemize}

%----------------------------------------------------------------------------------------
%	RELEVANT COURSE SECTION
%----------------------------------------------------------------------------------------
\section{Relevant \hspace{2mm} Courses}
Advance Computer Networks, Artificial Intelligence, Introduction to Machine Learining, Natural Language Processing, Convolutional Neural Networks for Visual Recognition 


%----------------------------------------------------------------------------------------
%	Selected Projects Section
%----------------------------------------------------------------------------------------
\section{Selected Projects}
All projects available on git : \url{https://www.github.com/akash9182}
\begin{itemize}[leftmargin=*]
\item \textbf{\href{https://github.com/akash9182/Product-recommendation-system}{Content based recommendation system}} : It is a flask-based REST webservice designed to be deployed to Heroku and relies on Anaconda for the scientific computing dependencies, and Redis to store precomputed similarities. A production-ready, content-based recommendation engine that computes similar items based on text descriptions.
\item \textbf{\href{https://github.com/akash9182/Text-generation}{Text generation}}: Built a LSTM recurrent neural network in TensorFlow that learns from Wikipedia text to generate new text. 
\item \textbf{\href{https://github.com/akash9182/BEGAN}{Generate Videos using BEGANS}}: It is an equilibrium enforcing method whose loss is derived from the Wasserstein distance. The model
balances the generator and discriminator and provides a new approximate convergence measure with stable training and high visual quality. Made contribution to the existing project by adding script to download the dataset and process the images.
\item \textbf{\href{https://github.com/akash9182/Pong-with-policy-gradinet}{How to Beat Pong using Policy Gradient}}: Using policy gradient technique from reinforcement learning to beat the game of Pong. Used OpenAI’s Universe as an environment for my agent.
\item \textbf{\href{https://github.com/akash9182/sentiment-analysis-mark1}{Sentiment-analysis-app}} : Android app that analyse your sentiment from text you write. I used TextBlob Python (2 and 3) library for processing textual data. It provides a simple API for diving into common natural language processing (NLP) tasks such as part-of-speech tagging, noun phrase extraction, sentiment analysis, classification, translation, and more. 
\item \textbf{\href{https://github.com/RiyanshKarani011235/handwriting-recognition-using-neural-networks-on-FPGA-final-year-project}{Real time Handwritten Character Recognition using Image processing and Neural Networks
implemented on an FPGA device}}Developed an inference model based on a feedforward Neural Network as an image classifier and
a grammar model trained with a large corpus of mathematical equations to translate real time video
feed of handwritten equations into a text stream. Implemented all image processing from scratch in
C, and implemented C-python binding to access the C Library from Python, and designed and
implemented a Neural Network classifier in Verilog for a Spartan 6 FPGA.
\item \textbf{{Chess Playing Articulated Robotic Arm}}
 : An articulated robotic arm controlled by resistive touch pad was designed with an intention to play
chess. It was controlled by an ATMega 128 and CoralDraw and Eagle softwares were used for links
and PCB designing. Matrix manipulation and Inverse Kinematics was used for transforming
coordinate system to inputs for motors.
\item \textbf{\href{https://github.com/projectscara2014/scara}{4-Axis SCARA robot}} : Worked on building the electronics, hardware and software for an interdisciplinary project involving
building a 4-axis SCARA robot.
\end{itemize}

%----------------------------------------------------------------------------------------
%	Courses
%----------------------------------------------------------------------------------------
\section{Online Courses}
\begin{itemize}[leftmargin=*]
 \item \href{http://web.stanford.edu/class/cs224n/}{\emph{Natural Language Processing, Stanford}} Learnt to implement, train, debug, visualize neural network models. On the model side, gained
understanding of word vector representations, window-based neural networks, recurrent neural
networks, long-short-term-memory models, recursive neural networks, convolutional neural networks
as well as some recent models involving a memory component. 
\item \href{http://www.kvpy.org.in/main/}{\emph{Machine Learning, Stanford University, Coursera}} Andrew Ng’s famous course, where I learnt in a more application oriented approach the basics of
Machine learning, and some simple Machine Learning models.
This course focused on the theoretical foundations of Statistical Learning Theory and some practical techniques
pertaining to Machine Learning including Validation and Regularisation.
\item \href{https://github.com/akash9182/My-Android-Nanodegree-Projects}{\emph{Android Nanodegree, Udacity}}
Learned to build android apps from scrtch in Andriod studio. Built some cool apps along the course.  
\end{itemize}


%----------------------------------------------------------------------------------------
%	HOBBIES SECTION
%----------------------------------------------------------------------------------------

\section{Hobbies}
Competitive Coding, Listneing to Audio Books, Badminton, Trekking
\end{resume}
\end{document}